\chapter{Experimental Testing}

\section{Hardware and Software setup}

Author used two devices for his investigation :
\begin{enumerate}
 \item Personal computer (PC) with Intel Core i5-7600 processor (Caby Lake architecure) with enabled hyperthreading and enabled SIMD support, 16 GB of HDDR4 HyperX RAM and 128 GB Kingston SSD.
  PC is operated by Windows 10 OS. ChakraCore was compiled with \texttt{mvsc 2017} compiler using both Debug and Release configurations.

\item Laptop MacBook Pro 2015 with Intel Core i5-4*** processor of Haswell series with enabled hyperthreading and SIMD support.
  MacBook has 8 GB of DDR3 RAM and 128 GB Apple SSD.
  It operates by MacOS 10.14.2. ChakraCore was compiled with clang 9.0 using test, debug and release builds.
\end{enumerate}

\section{Compile time results}

Compile time result will be compared between 5 different builds on two different platforms : Mac and Windows.

\addcontentsline{lot}{table}{Compile-time measures}
\begin{table}[]
\vspace{1.2cm}
\hspace{4.5cm}
\begin{tabular}{|l|l|l|l|}         \hline
Build: & Debug  & Release & Test   \\ \hline
MacOS: & 28 mins  & 22 mins & -    \\ \hline
PC & 7 mins & 5.5 mins  & 6.5 mins \\ \hline
\end{tabular}
\end{table}

\vspace{3cm}

\section{Run time tests and benchmarks}

Let's measure throughput of implemented algorithms : 

\begin{enumerate}
  
\item First benchmark is measurement and comparison of throughput of our \texttt{memory.copy} operation using MacBook. 
      It is copying one gigabyte of some memory filler with different ranges of data size and using different copy techniques.  
      Data size varies from 32 bytes to 1 megabyte. 
      Count of elements varies from 1024 (one chuck has 1 Mb) to 33 millions (chunk size is 32 bytes).
      Before the main measurement benchmark also has warmup phase, where ~100 megabytes of memory copies to other destination.
      The result is presented in table below. 

\small
\begin{table}[]
\hspace{2cm}
\addcontentsline{lot}{table}{Measurement of `memory.copy` operations and its rivals}
\begin{tabular}{|l|l|l|l|l|l|} \hline

 Size, count      & memory.copy & i64 x4 & i64 x2 & i32 x4 & i32 x2 \\ \hline
 32b, N=33554432  & 1.812       & 1.036  & 1.021  & 0.783  & 1.097  \\ \hline
 64b, N=16777216  & 3.459       & 1.670  & 1.580  & 1.068  & 1.733  \\ \hline
 128b, N=8388608  & 6.832       & 2.387  & 2.175  & 1.319  & 2.388  \\ \hline
 256b, N=4194304  & 13.724      & 3.013  & 2.675  & 1.434  & 2.731  \\ \hline
 512b, N=2097152  & 27.337      & 3.460  & 2.781  & 1.566  & 3.150  \\ \hline
 1.0Kb,N=1048576  & 55.630      & 2.888  & 2.532  & 1.277  & 2.802  \\ \hline
 2.0Kib, N=524288 & 111.520     & 3.798  & 3.191  & 1.390  & 3.629  \\ \hline
 4.0Kib, N=262144 & 223.065     & 3.858  & 3.343  & 1.334  & 3.606  \\ \hline
 8.0Kib, N=131072 & 445.831     & 3.966  & 3.280  & 1.485  & 3.697  \\ \hline
 16.0Kib, N=65536 & 884.956     & 4.006  & 2.806  & 1.658  & 2.705  \\ \hline
 32.0Kib, N=32768 & 1602.564    & 4.106  & 3.408  & 1.696  & 3.131  \\ \hline
 64.0Kib, N=16384 & 1795.975    & 2.958  & 2.711  & 1.677  & 3.719  \\ \hline
 128.0Kib, N=8192 & 1235.602    & 4.038  & 3.412  & 1.711  & 3.806  \\ \hline
 256.0Kib, N=4096 & 1333.333    & 4.105  & 3.423  & 1.665  & 3.166  \\ \hline
 512.0Kib, N=2048 & 277.778     & 3.078  & 2.927  & 1.272  & 2.842  \\ \hline
 1.0Mib, N=1024   & 344 .759    & 3.253  & 3.181  & 1.557  & 3.504  \\ \hline

\end{tabular}
\end{table}

\normalsize

\vspace{3cm}

\item Second benchmark is measurement and comparison of throughput of our \texttt{memory.fill} operation using MacBook. 
      It is storing one gigabyte of random values with different ranges of data size.  
      Data size varies from 32 bytes to 256 kbytes. 256 Kb) to 33 millions (chunk size is 32 bytes).
      Before the main measurement benchmark also has warmup phase, where ~100 megabytes of memory stores with random numeric value.
      The result is presented in table below.
\small
\begin{table}[]
\vspace{1cm}
\hspace{3cm}
\addcontentsline{lot}{table}{Measurement of `memory.fill` operations and its rivals}
\begin{tabular}{|l|l|l|l|} \hline

  Size, count       & memory.fill &   i64   &   f64   \\ \hline
  32b, N=33554432   & 337.541     & 231.047 & 202.857 \\ \hline
  64b, N=16777216   & 344.747     & 219.208 & 195.645 \\ \hline
  128b, N=8388608   & 309.110     & 236.186 & 200.928 \\ \hline
  256b, N=4194304   & 303.934     & 255.719 & 233.854 \\ \hline
  512b, N=2097152   & 316.871     & 235.361 & 209.014 \\ \hline
  1.0Kib, N=1048576 & 325.176     & 215.470 & 198.963 \\ \hline
  2.0Kib, N=524288  & 328.753     & 220.915 & 181.198 \\ \hline
  4.0Kib, N=262144  & 307.274     & 210.355 & 180.818 \\ \hline
  8.0Kib, N=131072  & 309.703     & 199.489 & 172.027 \\ \hline
  16.0Kib, N=65536  & 308.377     & 194.998 & 171.126 \\ \hline
  32.0Kib, N=32768  & 300.511     & 200.759 & 147.658 \\ \hline
  64.0Kib, N=16384  & 302.202     & 168.391 & 147.151 \\ \hline
  128.0Kib, N=8192  & 297.917     & 190.240 & 154.286 \\ \hline
  256.0Kib, N=4096  & 313.127     & 274.415 & 161.437 \\ \hline
\end{tabular}
\end{table}

\normalsize

\vspace{2cm}

\item Third benchmark is measurement and comparison of throughput of our \texttt{memory.init} operation using MacBook. 
      It is storing one gigabyte of random values with different ranges of data size during the passive segment initializing.  
      Data size varies from 32 bytes to 256 kbytes. 256 Kb) to 33 millions (chunk size is 32 bytes).
      Before the main measurement benchmark also has warmup phase, where ~100 megabytes memory is initializing with random numeric value.
      The result is presented in table below.
\small
\begin{table}[]
\hspace{3cm}
\addcontentsline{lot}{table}{Measurement of `memory.init` operations and its rivals}
\begin{tabular}{|l|l|l|l|} \hline

  Size, count       & memory.init &   i64   &   f64   \\ \hline
  32b, N=33554432   &   353.3     &  217.6  &  212.2   \\ \hline
  64b, N=16777216   &   313.1     &  211.1  &  203.2   \\ \hline
  128b, N=8388608   &   317.8     &  214.4  &  206.6   \\ \hline
  256b, N=4194304   &   320.0     &  221.2  &  232.2   \\ \hline
  512b, N=2097152   &   476.7     &  231.5  &  199.3   \\ \hline
  1.0Kib, N=1048576 &   407.5     &  355.8  &  274.9   \\ \hline
  2.0Kib, N=524288  &   365.9     &  229.1  &  199.4   \\ \hline
  4.0Kib, N=262144  &   581.2     &  317.7  &  249.2   \\ \hline
  8.0Kib, N=131072  &   337.0     &  217.6  &  210.6   \\ \hline
  16.0Kib, N=65536  &   313.3     &  249.9  &  286.3   \\ \hline
  32.0Kib, N=32768  &   304.0     &  230.7  &  257.7   \\ \hline
  64.0Kib, N=16384  &   411.2     &  212.8  &  233.0   \\ \hline
  128.0Kib, N=8192  &   308.6     &  214.7  &  207.8   \\ \hline
  256.0Kib, N=4096  &   320.6     &  204.6  &  192.5   \\ \hline
  512.0Kib, N=2048  &   282.7     &  163.9  &  155.1   \\ \hline

\end{tabular}
\end{table}

\normalsize

\end{enumerate}

\section{Conclusions}
