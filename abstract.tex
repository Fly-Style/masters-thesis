\chapter*{Abstract}
 
Bulk memory operations is a common memory calculation optimization technique that takes some memory pointer and make massive store to this memory. 
A lot of programs have intensive computations that use bulk operations like \texttt{memcpy} or \texttt{memmove} to improve copying or storing big chunks of memory. 
There are a lot of places where programmer can improve memory consumption and computation complexity of them application by increasing througput via bulk operations usage.

This thesis identifies bulk memory operations concept and implementation in WebAssembly virtual machine. 
It also presents approaches of optimtization connected with bulk operations : conditional segment initialization and passive segment initialiation. 
Bulk operations have their own intrinsics and any programmer will have a possibility to use them in their code. 
Additionally, any language-to-WebAssembly compiler can map its internal lanuguage bulk operations to a new WebAssembly bulk memory operations.   

All of the algorithms have been implemented in the Microsoft ChakraCore engine and tested on two devices : MacBook Pro 2015 with Retina equipped with a x86-64 processor (Haswell) 
and PC with Intel Core i5-7600 x86-64 processor and 16 GB of RAM using internal ChakraCore test suite, custom tests and benchmarks.  