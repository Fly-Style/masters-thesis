\chapter{WebAssembly}

\section{Abstract}

Nowadays browsers are the most popular application not only for surfing the Internet. 
There are a lot of office services (like word processors, spreadsheet editors, etc), games, video and audio streams, 3D editors are available in a browser; 
all of listed services require a good client performance for correct and fast work. 
Unfortunately, browsers cannot give such performance for this class of applications permanently.
Every web-application could be written as :
\begin{itemize}  
  \item typical JavaScript + HTML + CSS web-page with some features. 
        JavaScript has well-optimized just-in-time compiler with good startup and compilation time in every browser, but it's still not enough for maximum performance and safety.
  \item so-called 'rich' application, which requires special browser plugin and special applications for creating them. As a rule, they are executing in some external virutal machine. 
        It means that application has middle-valued throughput and big memory footprint, which could be minimized.
\end{itemize}

Performance challenge produced start of design and implementation of new low-level browser formal called WebAssembly.

\section{WebAssembly Overview} \cite{wasm1}
\subsection{General} \mbox{}

\indent WebAssembly (abbreviated WASM) is a safe, portable, low-level code format designed for efficient execution and compact representation. 
Its main goal is to enable high performance applications on the Web, but it does not make any Web-specific assumptions or provide Web-specific features, so it can be employed in other environments as well.

WebAssembly is an open standard developed by a W3C Community Group.

\subsection{Design Goals} \mbox{}

\indent The design goals of WebAssembly are the following:

\begin{itemize} 
  \item \textbf{Fast}: executes with near native code performance, taking advantage of capabilities common to all contemporary hardware.
  \item \textbf{Safe}: code is validated and executes in a memory-safe, sandboxed environment preventing data corruption or security breaches.
  \item \textbf{Well-defined}: fully and precisely defines valid programs and their behavior in a way that is easy to reason about informally and formally.
  \item \textbf{Hardware-independent}: can be compiled on all modern architectures, desktop or mobile devices and embedded systems alike.
  \item \textbf{Language-independent}: does not privilege any particular language, programming model, or object model.
  \item \textbf{Platform-independent}: can be embedded in browsers, run as a stand-alone VM, or integrated in other environments.
  \item \textbf{Open}: programs can interoperate with their environment in a simple and universal manner.
  \item \textbf{Compact}: has a binary format that is fast to transmit by being smaller than typical text or native code formats.
  \item \textbf{Modular}: programs can be split up in smaller parts that can be transmitted, cached, and consumed separately.
  \item \textbf{Efficient}: can be decoded, validated, and compiled in a fast single pass, equally with either just-in-time (JIT) or ahead-of-time (AOT) compilation.
  \item \textbf{Streamable}: allows decoding, validation, and compilation to begin as soon as possible, before all data has been seen.
  \item \textbf{Parallelizable}: allows decoding, validation, and compilation to be split into many independent parallel tasks.
  \item \textbf{Portable}: makes no architectural assumptions that are not broadly supported across modern hardware. WebAssembly code is also intended to be easy to inspect and debug, especially in environments like web browsers, but such features are beyond the scope of this specification.
\end{itemize}

At its core, WebAssembly is a virtual instruction set architecture (virtual ISA). As such, it has many use cases and can be embedded in many different environments.

\subsection{Concepts} \cite{wasm1}

WebAssembly encodes a low-level, assembly-like programming language. 
This language is structured around the following concepts:

\begin{itemize} 
  
  \item \textbf{Values}:
  WebAssembly provides only four basic value types. 
  These are integers and IEEE 754-2008 numbers, each in 32 and 64 bit width. 32 bit integers also serve as Booleans and as memory addresses. The usual operations on these types are available, including the full matrix of conversions between them. There is no distinction between signed and unsigned integer types. Instead, integers are interpreted by respective operations as either unsigned or signed in two’s complement representation.
  
  \item \textbf{Instructions}:
  the computational model of WebAssembly is based on a stack machine. 
  Code consists of sequences of instructions that are executed in order. 
  Instructions manipulate values on an implicit operand stack * and fall into two main categories. 
  Simple instructions perform basic operations on data. They pop arguments from the operand stack and push results back to it. 
  Control instructions alter control flow. 
  Control flow is structured, meaning it is expressed with well-nested constructs such as blocks, loops, and conditionals. 
  Branches can only target such constructs.
  
  \item \textbf{Traps}:
  under some conditions, certain instructions may produce a trap, which immediately aborts execution. 
  Traps cannot be handled by WebAssembly code, but are reported to the outside environment, where they typically can be caught.
  
  \item \textbf{Functions}:
  code is organized into separate functions. 
  Each function takes a sequence of values as parameters and returns a sequence of values as results.** 
  Functions can call each other, including recursively, resulting in an implicit call stack that cannot be accessed directly. 
  Functions may also declare mutable local variables that are usable as virtual registers.
  
  \item \textbf{Tables}:
  a table is an array of opaque values of a particular element type. 
  It allows programs to select such values indirectly through a dynamic index operand. 
  Currently, the only available element type is an untyped function reference. 
  Thereby, a program can call functions indirectly through a dynamic index into a table. 
  For example, this allows emulating function pointers by way of table indices.
  
  \item \textbf{Linear Memory}:
  a linear memory is a contiguous, mutable array of raw bytes. 
  Such a memory is created with an initial size but can be grown dynamically. 
  A program can load and store values from/to a linear memory at any byte address (including unaligned). 
  Integer loads and stores can specify a storage size which is smaller than the size of the respective value type. 
  A trap occurs if an access is not within the bounds of the current memory size.
  
  \item \textbf{Modules}:
  a WebAssembly binary takes the form of a module that contains definitions for functions, tables, and linear memories, as well as mutable or immutable global variables. 
  Definitions can also be imported, specifying a module/name pair and a suitable type. 
  Each definition can optionally be exported under one or more names. 
  In addition to definitions, modules can define initialization data for their memories or tables that takes the form of segments copied to given offsets. 
  They can also define a start function that is automatically executed.
  
  \item \textbf{Embedder}:
  a WebAssembly implementation will typically be embedded into a host environment. 
  This environment defines how loading of modules is initiated, how imports are provided (including host-side definitions), and how exports can be accessed. 
  However, the details of any particular embedding are beyond the scope of this specification, and will instead be provided by complementary, environment-specific API definitions.
  
  *	 In practice, implementations need not maintain an actual operand stack. Instead, the stack can be viewed as a set of anonymous registers that are implicitly referenced by instructions. The type system ensures that the stack height, and thus any referenced register, is always known statically.
  
  ** In the current version of WebAssembly, there may be at most one result value.
\end{itemize}

\subsection{Semantic Phases} \mbox{}

\indent Conceptually, the semantics of WebAssembly is divided into three phases. For each part of the language, the specification specifies each of them.

% \begin{itemize}
\textbf{Decoding}. WebAssembly modules are distributed in a binary format. 
                   Decoding processes that format and converts it into an internal representation of a module. 
                   In this specification, this representation is modelled by abstract syntax, but a real implementation could compile directly to machine code instead.
  
\textbf{Validation}. A decoded module has to be valid. 
                     Validation checks a number of well-formedness conditions to guarantee that the module is meaningful and safe. In particular, it performs type checking of functions and the instruction sequences in their bodies, ensuring for example that the operand stack is used consistently.

\textbf{Execution}. Finally, a valid module can be executed. Execution can be further divided into two phases:

\textbf{Instantiation}. A module instance is the dynamic representation of a module, complete with its own state and execution stack. 
                        Instantiation executes the module body itself, given definitions for all its imports. 
                        It initializes globals, memories and tables and invokes the module’s start function if defined. 
                        It returns the instances of the module’s exports.
\textbf{Invocation}. Once instantiated, further WebAssembly computations can be initiated by invoking an exported function on a module instance. 
                     Given the required arguments, that executes the respective function and returns its results.

Instantiation and invocation are operations within the embedding environment.
% \end{itemize}

\subsection{Enhancement process}
WebAssembly community has big growth and they actively help to improve this technology.
There are two divided community groups : \cite{wasm_wg} \cite{wasm_cg}
\begin{itemize}
  \item \textbf{Working Group.}  The scope of the WebAssembly Working Group comprises addressing the need for native-performance code on the Web in applications ranging from 3D games to speech recognition to codecs—and in any other contexts in which a common mechanism for enabling high-performance code is relevant—by providing a standardized portable, size-, and load-time-efficient format and execution environment that attempts to maximize performance and interoperate gracefully with JavaScript and the Web, while ensuring security and consistent behavior across a variety of implementations.
  
  \item \textbf{Community Group.} The mission of Community Group is to provide a forum for pre-standardization collaboration on the WebAssembly format.
        The Community Group is explicitly not intended to standardize the final ratified text of standards related to WebAssembly. 
        Instead it will prepare recommendations to a separate WebAssembly Working Group.
\end{itemize}

There are a lot of community proposals inside the WebAssembly enhancement process. Let's count the most important proposals :
\begin{itemize}
\item \textbf{Reference types}. WebAssembly still doesn't include reference types. The key idea of this proposal is to support
      \texttt{nullref} and give a possibility to manipulate a tables inside WebAssembly memory.
\item \textbf{Garbage collection}. WebAssembly doesn't have automatic memory management system, which can be build after implementation of reference types. It gives a possibility to count references or trace whole 'heap', staring from the current rootset. 
\item \textbf{Bulk memory operations}. 
\item \textbf{Atomics and SIMD operations}. There are a lot of processors which have special registers for vectors computation
  (or SIMD - single instruction, multiply data). All the most popular architectures have SIMD extensions.
   For example, Intel has AVX and SSE registers with 128 to 512 bytes capacity.
   Arm has NEON, PowerPC have AltiVEC and SPE, MIPS have MaDMaX technologies for SIMD operations.
   Of course, it is a great challenge to support SIMD operations both for compilers and for runtime, because it can improve the performanc of some complex computations in a times.
\end{itemize}
  
\section{Overview of Rivals}
\subsection{Java Applets} \mbox{}\\
\indent A Java applet is a small application is written in the Java language, or another programming language that compiles to Java bytecode (JVM intermediate representation), and delivered to users in the form of Java bytecode. 
The browser user launched the Java applet from a web page, and the applet was then executed within JVM were launched in separate process from the web browser itself. 
Programmers could appear a Java applets in any frame of the web page, a new application window, Sun's AppletViewer, or an another stand-alone tool.

Java applets had access to hardware acceleration, making them well-suited for non-trivial, computation-intensive visualizations. When applets were popular, JavaScript engines didn't have GPU support.
As browsers have gained support for hardware-accelerated graphics thanks to the canvas technology (or specifically WebGL in the case of 3D graphics), 
as well as just-in-time compiled JavaScript became very efficient, the speed difference has become less noticeable.

The applet can be displayed on the web page by making use of the deprecated applet HTML element, or the recommended object element.
The embed element can be used with Mozilla family browsers (embed was deprecated in HTML 4 but is included in HTML 5). 
This specifies the applet's source and location. 
Both object and embed tags can also download and install Java virtual machine (if required) or at least lead to the plugin page. 
\texttt{applet} and \texttt{{object}} tags also support loading of the serialized applets that start in some particular (rather than initial) state. 
Tags also specify the message that shows up in place of the applet if the browser cannot run it due to any reason \cite{Wiki_Applet}.

Java applets were deprecated since JDK 9 in 2017 and removed from JDK 11, released in September 2018.

\textit{\textbf{By definition, Java bytecode is platform independent intermediate representation, Java applets can be executed by browsers (or other clients) for many platforms, including Microsoft Windows, FreeBSD, Unix, macOS and Linux.}}
\textit{\textbf{They cannot be run on modern mobile devices, which do not support Java.}}

\subsection{Microsoft Silverlight} \cite{Wiki_Silver}

Microsoft Silverlight (or simply Silverlight) was an application framework for writing and running rich Internet applications, similar to Adobe Flash. 
A plugin for Silverlight is still available for some browsers. 
While early versions of Silverlight focused on streaming media, later versions supported multimedia, graphics, and animation and gave developers support for CLI languages and development tools.

Silverlight provides a retained mode graphics system similar to Windows Presentation Foundation (WPF), and integrates multimedia, graphics, animations, and interactivity into a single run-time environment. 
In Silverlight applications, user interfaces are declared in Extensible Application Markup Language (XAML) and programmed using a subset of the .NET Framework. 
XAML can be used for marking up the vector graphics and animations.

Silverlight supports H.264 video, Advanced Audio Coding, Windows Media Video (WMV), Windows Media Audio (WMA), and MPEG Layer III (MP3) 
media content across all supported browsers without requiring Windows Media Player, the Windows Media Player ActiveX control, or Windows Media browser plug-ins. 
Silverlight exposes a Downloader object which can be used to download content, like scripts, media assets, or other data, as may be required by the application.
With version II, the programming logic can be written in any .NET language, including some derivatives of common dynamic programming languages like \textbf{IronRuby} and \textbf{IronPython}.

Cross platform Mozilla Firefox support for Silverlight was removed in Firefox 52 released in March 2017 when Mozilla removed support for NPAPI plugins,[40] bringing it in-line with the removal of NPAPI plugin support in Google Chrome.

\subsection{Adobe Flash} \cite{Wiki_Flash}

Adobe Flash Player is computer software for using content created on the Adobe Flash platform, including viewing multimedia contents, executing rich Internet applications, and streaming audio and video. 
Flash Player can run from a web browser as a browser plug-in or on supported mobile devices. 
Flash Player was created by Macromedia and has been developed and distributed by Adobe Systems since Adobe acquired Macromedia in 2005. 

Flash Player runs \textbf{SWF} files that can be created by Adobe Flash Professional, Adobe Flash Builder or by third party tools such as FlashDevelop. 
Flash Player supports vector graphics, 3D graphics, embedded audio, video and raster graphics, and a scripting language called \textbf{ActionScript}. 
ActionScript is based on ECMAScript (similar to JavaScript) and supports object-oriented code. 
Flash Player is distributed free of charge and its plug-in versions are available for every major web browser and operating system. 
Google Chrome, Internet Explorer 11 in Windows 8 and later, and Microsoft Edge come bundled with a sandboxed Adobe Flash plug-in.

Flash Player once had a large user base, and was a common format for web games, animations, and graphical user interface (GUI) elements embedded in web pages. 
Adobe stated in 2013 that more than 400 million out of over 1 billion connected desktops update to the new version of Flash Player within six weeks of release. 

Flash Player has become increasingly criticized for its performance, consumption of battery on mobile devices, the number of security vulnerabilities that had been discovered in the software, and its closed platform nature.

In July 2017, Adobe announced that it would end support for Flash Player in 2020, and continued to encourage the use of open HTML5 standards in place of Flash. 
The announcement was coordinated with all tech giantes.
Its usage has also waned because of modern web standards that allow some of Flash's use cases to be fulfilled without third-party plugins.

\subsection{asm.js} \cite{asm17}

\texttt{asm.js} a very small, strict subset of JavaScript that only allows things like \texttt{while}, \texttt{if}, numbers, top-level named functions, and other simple constructs. 
It does not allow objects, strings, closures, and basically anything that requires heap allocation. 
Asm.js code resembles C in many ways, but it's still completely valid JavaScript that will run in all current engines. 
It pushes JS engines to optimize this kind of code, and gives compilers like Emscripten a clear definition of what kind of code to generate. 
We will show what \texttt{asm.js} code looks like and explain how it helps and how you can use it.

This subset of JavaScript is already highly optimized in many JavaScript engines using fancy Just-In-Time (JIT) compiling techniques. 
However, by defining an explicit standard we can work on optimizing this kind of code even more and getting as much performance as we can out of it. 
It makes it easier to collaborate across multiple JS engines because it's easy to talk about and benchmark. 
The idea is that this kind of code should run very fast in each engine, and if it doesn't, it's a bug and there's a clear spec that engines should optimize for.

It also makes it easy for people writing compilers that want to generate high-performant code on the web. 
They can consult the \texttt{asm.js} spec and know that it will run fast if they adhere to \texttt{asm.js} patterns. 
Emscripten, a C/C++ to JavaScript compiler, emits \texttt{asm.js} code to make it run with near native performance on several browsers.

Additionally, if an engine chooses to specially recognize asm.js code, there even more optimizations that can be made. 
Firefox is the only browser to do this right now.

\subsection{Comparison WebAssembly with Rivals}
\subsubsection{WebAssembly vs rich applications}\mbox{}\\
So-called 'rich' application is an application that is executing on browser plugin which supports programming language, unlike WebAssembly.
WebAssembly application user doesn't require any additional plugins, this is native browser technology.

We can summarize all advantages in list of WebAssembly's major features which don't available in 'rich' applications :
\begin{itemize}  
  \item \textbf{Performance}. Designed as low-level binary instruction set, it provides high speed of parsing and execution which is higher than execution of 'rich' application. 
                Also, it uses SIMD instructions, atomic operations. Threads and GC are intended to implement in WASM engines. 
  \item \textbf{Security}. WebAssembly’s applications and JS in browsers use same API for communication with hardware and OS.
  \item \textbf{JS compatibility with WebAssembly}.
  \item \textbf{Language-independent}. It does not privilege any particular language, programming model, or object model, unlike in 'rich' applications, 
                which depend from concrete target virtual machine or programming language.
  \item \textbf{Compact}. WebAssembly has a binary format which is smaller than typical text or native code formats.
  \item \textbf{Open}. WebAssembly is open format with community-driven specification, unlike 'rich' applications which have vendor-locked specification.

\end{itemize}

\subsubsection{WebAssembly vs asm.js}\mbox{}\\
\indent As it was defined higher, \texttt{asm.js} is just a subset of JavaScript language, which gives a possibility to write more performant code using types and low-level data manipulations.
I assume that \texttt{asm.js}'s has that biggest problems: highly increased code complexity (complex developer experience) and small browser support (Firefox).

We can summarize all advantages of WebAssembly over \texttt{asm.js} \cite{wasm_asm}:

\textbf{1. Startup.}

WebAssembly is designed to be small to download and fast to parse, so that even large applications start up quickly.

It’s actually not that easy to improve on the download size of gzipped minified JavaScript, as it’s already fairly compact when compared with native code. Still, WebAssembly’s binary format can improve on that, by being carefully designed for size in mind (indexes are LEB128s, etc.). It is often around 10–20 percent smaller (comparing gzipped sizes).

WebAssembly improves on parsing in a much bigger way: It can be parsed an order of magnitude faster than JavaScript. This mostly comes down to binary formats being faster to parse, especially ones designed for that. WebAssembly also makes it easy to parse (and optimize) functions in parallel, which helps a lot on multicore machines.

Total startup time can include factors other than downloading and parsing, such as the VM fully optimizing the code, or downloading additional data files that are necessary before execution, etc. But downloading and parsing are unavoidable and therefore important to improve upon as much as possible. All the rest can be optimized or mitigated, either in the browser or in the app (for example, fully optimizing the code can be avoided by using a baseline compiler or interpreter for WebAssembly, for the first few frames).

\textbf{2. CPU features.}

One trick that’s made \texttt{asm.js} so fast is that while all JavaScript numbers are doubles, in \texttt{asm.js} an addition will have a bitwise-and operation right after it, which makes it logically equivalent to the CPU doing a simple integer addition, which CPUs are very good at. So \texttt{asm.js} made it easy for VMs to use a lot of the full power of CPUs.

But \texttt{asm.js} was limited to things that are expressible in JavaScript. WebAssembly isn’t limited in that way, and lets us use even more CPU features, such as:
\begin{itemize}
\item \textbf{64-bit integers}. Operations on them can be up to 4x faster. This can speed up hashing and encryption algorithms, for example.
\item \textbf{Load and store offsets}. This helps very broadly, basically anything that uses memory objects with fields at fixed offsets (C structs, etc.).
\item Unaligned loads and stores, avoiding asm.js’s need to mask (which \texttt{asm.js} did for Typed Array compatibility purposes). 
      This helps with practically every load and store.
\item Various CPU instructions like popcount, copysign, etc. Each of these can help in specific circumstances (e.g. popcount can help in cryptanalysis).
\item How much a specific benchmark benefits will depend on whether it uses the features mentioned above. 
      We often see a 5 percent speedup on average compared to \texttt{asm.js}. Further speedups are expected in the future from CPU features like SIMD.
\end{itemize}
\textbf{3. Toolchain Improvements.}

WebAssembly is primarily a compiler target, and therefore has two parts: Compilers that generate it (the toolchain side), and VMs that run it (the browser side). Good performance depends on both.

This was already the case with \texttt{asm.js}, and Emscripten did a bunch of toolchain optimizations, running LLVM’s optimizer and also Emscripten’s asm.js optimizer. For WebAssembly, we built on top of that, but have also added some significant improvements while doing so. Both asm.js and WebAssembly are not typical compiler targets, and in similar ways, so lessons learned during the asm.js days helped do things better for WebAssembly. In particular:
\begin{itemize}
\item Mozilla replaced the Emscripten \texttt{asm.js} optimizer with the Binaryen WebAssembly optimizer, which is designed for speed. That speed lets us run more costly optimization passes. For example, they removed duplicate functions by default when optimizing, which often shrinks large compiled C++ codebases by around 5 percent.
\item Better optimizations for irreducible and convoluted control flow, improving the Relooper algorithm.
\item Overall, these toolchain improvements help about as much as moving from \texttt{asm.js} to WebAssembly.
\end{itemize}
  
\textbf{4. Predictably Good Performance.}

\texttt{asm.js} could run at basically native speed, but it never actually did so in all browsers consistently. The reason is that some tried to optimize it one way, some another, with differing results. Over time things started to converge, but the basic problem was that asm.js was not an actual standard: It was an informal spec of a subset of JavaScript, written by one vendor, that only gradually saw interest and adoption from the others.  
