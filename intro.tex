\chapter*{Introduction}

There are near 4 billion Internet users in the world today. The very big part of them are using browsers for surfing the Internet.
It makes browsers the most popular software application in the world (except operating systems) and it means that overall time of code execution and web-page loading should be minimized. There is an investigation of 'USA Today' which shows that :

\begin{itemize}
 \item Amazon could lose 1.6 billion dollars per year if their web-page delay increase by 1 second;
 \item Slowing Google's search result by 4/10's of second would reduce number of searches by 8 million in a day;
 \item One of four web-site visitors would abandon this page if it takes more than a second to load it;
 \item One of five web-site visitors become rude to web-service because they 'serving them too slowly';
\end{itemize}

All these facts mean one thing - it’s profitable both for browser manufacturers(which are the biggest serivces providers) and for developers community to invest in developing and improving the productivity of websites and web applications.

WebAssembly is of the major result of mutual research and design of performant Web tools. WebAssembly was  was first announced 17th of June 2015, and the first demonstration was executing Angry Bots in three major browsers.
The precursor technologies were asm.js from Mozilla and Google Native Client (PNaCl) and the initial implementation was based on the feature set of \texttt{asm.js}. \cite{Wiki_Wasm}

WebAssembly (as a rule, this term is shortened to WASM) is a standard that defines a portable binary format and a corresponding assembly-like text format for executables. The main goal of the format is to enable high performance applications on web pages, 
but it is designed to be executed and integrated in other environments as well, not only in Web-applications. \cite{Wiki_Wasm}

\textbf{Original memory management in WebAssembly.}

Sometimes software engineers want to copy or initialize big chunk of linear memory.

For this reason they are using \texttt{memset/memcpy} in C/C++ or \texttt{System.arraycopy()} in Java, generally called as \textbf{bulk memory operations}. 

\begin{verbatim}
#include <string.h>
int testFunction() {
  size_t length = 1024;           // 2^10 for test purposes 
  int *array1 = (int*) malloc(length * sizeof(array1));
  // ... some usage of this mempory here
  int *array2 = (int*) malloc(length * sizeof(array2));
  
  for(int i = 0; i < length; ++i)
    array1[i]++;                  // avoid constant propagation. 
  memcpy(array2, array1, length); // copy data from array1 to array2 
  memset(array1, 0, len);         // filling all array1 memory by zeroes 
  return array1[0] + array2[0];   // avoid dead code elimination.
}
\end{verbatim}

Compilation result of this fucntion by \texttt{emscripten} tool to such piece of code (full code is available in Appendix A) :
\small
\begin{verbatim}
(module
 (import "env" "memset" (func $memset (param i32 i32 i32) (result i32)))
 (import "env" "memcpy" (func $memcpy (param i32 i32 i32) (result i32)))
 (table 0 anyfunc)
 (export "memory" (memory $0))
 (export "_Z12testFunctionPi" (func $_Z12testFunctionPi))

 // compiled testFunction.
 (func $_Z12testFunctionPi (; 3 ;) (param $0 i32) (result i32)
  (local $1 i32)  (local $2 i32) (local $3 i32) (local $4 i32)
  (set_local $1
   (call $memset          // <- memset call
    (call $_Znaj
     (i32.const 4096)
    )
    (i32.const 10)
    (i32.const 1024)
   )
  )
  // ... a lot of generated code with folded constants.
  (i32.load
   (call $memcpy          // <- memcpy call
    (get_local $2)
    (get_local $1)
    (i32.const 1024)
   )
  )
 ))
)
\end{verbatim}

\normalsize

Compiled code contains \texttt{memcpy} and \texttt{memset} functions.
They were imported from \texttt{env} module which \textbf{\textit{is linked}} with standart \texttt{libc} library, where those functions are defined and implemented.
It would be better to use internal WebAssembly bulk operations instead of using \texttt{libc}'s memory operations. 

This is a key idea of further investigations : desiging and implementation of bulk memory operations intrinsics, which would be placed to a special \texttt{memory} module.


\textbf{Summary of Major Contributions.}

The major contributions of this thesis are:
\begin{itemize}
  \item Description of algorithms of bulk copy, move and initialization operations on WebAssembly memory.
  \item Development of passive memory initialization algorithms.
  \item Implementation of bulk operations algorithms.
  \item Integrations of algorithms to a production virtual machine named ChakraCore.
  \item Study and comparison of the performance of these algorithms compare to standard.
\end{itemize}

The algorithms have been implemented in C++ and integrated to the ChakraCore engine.
They can be used as comilation result of any compiled user's WebAssembly module.
These algorithms are fully compatible with WebAssembly specification and bulk memory operations proposal.

Chapter 1 provides introduction about WebAssembly as technology and its ancestors. 
Chapter 2 describes the major WebAssembly compilers and runtimes.
Сhapter 3 presents the developed algorithms for every bulk operation and their profitable usage. 
The results of compilation, runtime and benchmarks tests are presented in Chapter 4.
Appendix A contains an example of WebAssembly S-expressions.