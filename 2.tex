\chapter{WebAssembly Compilers and Engines Overview}

As any other programming language ecosystem, WebAssembly has source to WAT or source to WASM compilers and IR interpreters (or runtimes).
The most used source languages for WebAssembly are C/C++ and Rust, which are used for development of high-performant applications.
Also JavaScript and TypeScript are popular for compiling to WebAssembly IR or binary code.
This chapter contains short overview of the most efficient tools for WebAssembly compilation and execution.

\section{Compilers}

\subsection{emscripten}  \mbox{}

\indent \texttt{emscripten} is the most popular and the most efficient LLVM to Web ahead-of-time compiler and debugger tools.
It supports all languages which could be compiled to LLVM bitcode.

Originally, it had only JavaScript as a compilation target language, but in version \texttt{1.39},
\texttt{emscripten} develpers shipped new \texttt{binaryen} compiler infrastructure as part of \texttt{emscripten} tool.
It allows fast, effective and easy compilation :
\begin{itemize}
 \item \texttt{binaryen} has a single C header, where whole \texttt{C API} is described. This header could be used from JavaScript.
 \item \texttt{binaryen} designed as multicore tools for completely parallen code generation and optimization techniques.
 \item as a rule, \texttt{binaryen} compiler make a lot of compilation passes for better optimization.
   It uses both common and specific compilers optimization techniques as constant folding, loop fusion,
   dead code elimination, experimental escape analysis and target code minification.
 \item \texttt{binaryen} also includes WebAssembly interpreter and separate WebAssmbly IR optimizer. 
\end{itemize}

Nowadays (year 2019) \texttt{emscripten} compiler infrastructure is the most popular among C/C++/Swift developers.

\subsection{Rust WebAssembly target} \mbox{}

\indent  Rust compiler infrastructure supports Rust to WAT and Rust to WASM since version \texttt{1.30}.
WAT and WASM are one of the backend output options.
You can add the target by executing \texttt{rustup target add wasm32-unknown-unknown} in your terminal.
Compilation of Rust program is possible by executing \texttt{cargo build --target wasm32-unknown-unknown --release}.

Rust optimized compiler has big featureset and it is one of the most efficient compiler nowadays.

\subsection{Go WebAssembly target} \mbox{}

Go compiler infrastructure supports Go to WASM experimental compilation since version \texttt{1.11}.

\subsection{AssemblyScript} \mbox{}

\indent AssemblyScript is TypeScript-to-WASM compiler was written in TypeScript.
It has its own compiler frontend and uses \texttt{binaryen} as a compiler backend. 
AssemblyScript team has a lot of demos and tutorials how to start working with AssemblyScript.
Author of this thesis was used AssemblyScript compiler for compilation his own runtime test and benchmarks.

\subsection{TeaVM}

TeaVM is an ahead-of-time compiler for Java bytecode that emits JavaScript or WebAssembly that runs in a browser.
Its close relative is the well-known GWT.
The biggest difference is that TeaVM can use compiled Java class files.
Moreover, the source code is not required to be Java, so TeaVM successfully compiles Kotlin and Scala. \cite{wasm_tea}
Also, TeaVM produces fast, small JavaScript code for web apps that start quickly, even on mobile devices.

TeaVM tries to reconstruct original structure of a method, so in most cases it produces JavaScript that developers would write manually.
TeaVM has a very sophisticated optimizer, which knows a lot about the code. Some examples are:
\begin{itemize}
\item Dead code elimination produces very small JavaScript.
\item Devirtualization turns virtual calls into static function calls, which makes code faster.
\item TeaVM can reuse one local variables to store several local variables.
\item TeaVM renames methods to as short forms as possible; UglifyJS usually can’t perform such optimization.
\item TeaVM supports threads. TeaVM is capable of transforming methods to continuation-passing style.
  This makes possible to emulate multiple logical threads in one physical thread. TeaVM threads are, in fact, green threads.
\end{itemize}

\section{Runtimes}

The second part of WebAssembly ecosystem is runtimes.
They are presented by browsers execution environments and WebAssembly VM for embedded and research. 

\subsection{Reference Interpreter} \mbox{}

WebAssembly specification contains a link to so-called \textit{reference interpreter}.
This is a tool which both W3C Working Group and Community Group are using for development of a new features prototypes.
The reference interpreter is located by link \cite{wasm_ref}.
Author of this thesis was used this interpreter to compare his implementation with the specification.

\subsection{JavaScript virtual machines} \mbox{}

This section contains description of virtual machines which support WebAssembly binaries execution.
Following the statistics, they are the platform with the highest usage of WebAssembly context.

\subsubsection{V8} \mbox{}

Chrome V8, or V8, is a JavaScript engine developed by dutch Google office for Google Chrome and Chromium browsers. It is written on C++.
The first version of the V8 engine was released September 2, 2008, the same date when first version of Google Chrome was released.
It has also been used as query execution context in different NoSQL databases,
   in Node.js, which is JS exection plarform on the server-side and in Electron framework.

V8 has Ignition interpreter, which executes sources after VM initialization and collects heuristics for TurboFan JIT-compiler,
   which compiles JavaScript directly to native machine code, based on collected statistics.
The compiled code is optimized dynamically at runtime, based on heuristics of the code's execution profile.
Optimization techniques used include inlining, polymorphic inline caching and elision of expensive runtime properties;
   it also provides escape analysis and scalar replacement based on sea-of-nodes IR techniques.
V8 contains a generational incremental collector and new experimental parallel and gerenational garbage collector called Orinoco.
This engine uses Torque language for new features implementation within.  

V8 can compile sources to x86 PowerPC, ARM and MIPS instructions set. \cite{wiki_v8}

Google Chrome supports WebAssembly 1.0 since version 57 (V8 version 5.7).
Nowadays V8 has the most complete WebAssembly featureset including implementation level proposals.

\subsubsection{SpiderMonkey} \mbox{}

SpiderMonkey is a JavaScript engine developed by Brendan Eich in Netscape Communications as first JavaScript execution engine. 
SpiderMonkey provides JavaScript support for Mozilla Firefox and various embeddings such as the GNOME 3 desktop, Yahoo Widgets or Adobe Acrobat.

SpiderMonkey has one interpreter which executes sources and collects heuristics during VM startupand two JIT-compilers : baseline and IonMonkey.
They have different optimization levels and they are designed for better balance between compilation time
   (IonMonkey produces highly optimized code, but spends more time than baseline compiler) and exectuion time.
Baseline compiler uses optimizations like polymorphic inline caches, inlining, proxies, minor gc calls and
   common optimization techniques like constant folding etc.

SpiderMonkey has a mark-sweep-compact generational incremental garbage collection. Much of the collector work is performed on helper threads.

SpiderMonkey engine supports IA-32, x86-64, ARM, MIPS, SPARC architectures.

Mozilla Firefox supports WebAssembly 1.0 since version 52. SpiderMonkey implements all specifications features and some high-level proposals.

\subsubsection{JSCore} \mbox{}

JavaScriptCore (simply JSCore) is a JavaScript engine developed in Apple. 
JavaScriptCore provides JavaScript support for Safari browser and various browsers based on WebKit engine and Steam (game delivery service from Valve).

JavaScriptCore has one interpreter which executes sources and collects heuristics during VM startup and three JIT-compilers : baseline, DFG and FTL.
As SpiderMonkey, they have different optimization levels and they are designed for better balance between compilation time and exectuion time.
Some experts think that JavaScriptCore has the most successful and promising architecture between all 4 major browsers.

JavaScriptCore contains a mark-compact generational garbage collection. Much of the collector work is performed on helper threads.

JavaScriptCore engine supports x86-64 and ARM architectures.

Safari supports WebAssembly 1.0 since version 11. Safari implements all specifications features and some high-level proposals.

\subsubsection{ChakraCore} \mbox{}

ChakraCore is a JavaScript engine developed in Microsoft. 
JavaScriptCore provided JavaScript support for Internet Explorer, Microsoft Edge browsers, Node.js-chakra and different databases engine.
It is popular for embedded system due to extended support.

ChakraCore has one interpreter which executes sources and collects heuristics during VM startup and two JIT-compilers : SimpleJIT and FullJIT.
They have different optimization levels and they are designed for better balance between compilation time and exectuion time.
ChakraCore contains a mark-sweep generational garbage collection called \texttt{Recycler}. Much of the collector work is performed on helper threads.
Also it has very interesting allocations algorithms called \texttt{ArenaAllocator}.

JavaScriptCore engine supports i386, x86-64 and ARM architectures.

Edge supports WebAssembly 1.0 since version 16, in return Internet Explorer doesn't suport WebAssembly.
ChakraCore support only specifications features and a few important proposals, like atomics and threading.
In February, 2019 Mirosoft announces that they would build next Edge browser versions based on Chrominum, but they will continue support of ChakraCore due to big popularity in embedded industry.

Author uses ChakraCore for creation of prototype of bulk operations implementation.

\subsection{Other runtimes}

There are also a couple of other runtimes and virtual machines which can compile and/or execute WebAssembly binaries.
As a rule, such WebAssembly runtimes which pedantically follows the WebAssembly specification.
They was created by enthusiasts who want to train them hard skills or who cares about WebAssemly as technology.
